\documentclass{article}
\usepackage{fancyhdr}
\usepackage{amsthm}
\usepackage{etoolbox}
\usepackage{verbatim}
\usepackage{enumerate}
\usepackage{amsmath}
\usepackage{algorithmicx}
\usepackage{algorithm}
\usepackage{algpseudocode}
\usepackage{amssymb}
\usepackage{tikz}
	
\pagestyle{fancy}
\title{Chapter 26}
\author{Michelle Bodnar, Andrew Lohr}

\newcounter{curnum}
\setcounter{curnum}{0}

\newtheorem{th1}{Exercise} 
\newcommand{\calH}{\mathcal{H}}
\newcommand{\calX}{\mathcal{X}}
\newcommand{\calA}{\mathcal{A}}
\newcommand{\calY}{\mathcal{Y}}

\begin{document}
\maketitle
\noindent\textbf{Exercise 26.1-1}\\
To see that the networks have the same maximum flow, we will show that every flow through one of the networks corresponds to a flow through the other. First, suppose that we have some flow through the unsplit network. Since we are only changing one of any pair of antisymmetric edges, for any edge that is unchanged by the splitting, we just have an identical flow going through those edges. Suppose that there was some edge $(u,v)$ that was split because it had an anitsymmetric edge, and we had some flow, $f(u,v)$ in the original graph. Since the capacity of both of the two edges that are introduced by the splitting of that edge have the same capacity, we can set $f'(u,v) = f'(u,x) = f'(x,v)$. By constructing the new flow in this manor, we have an identical total flow, and we also still have a valid flow.

Similarly, suppose that we had some flow $f'$ on the graph with split edges, then, for any triple of vertices $u,x,v$ that correspond to a split edge, we must have that $f'(u,x) = f'(x,v)$ because the only edge into $x$ is $(u,x)$ and the only edge out of $x$ is $(x,v)$, and the net flow into and out of each vertex must be zero. We can then just set the flow on the unsplit edge equal to the common value that the flows on $(u,x)$ and $(x,v)$ have. Again, since we handle this on an edge by edge basis, and each substitution of edges maintains the fact that it is a flow of the same total, we have that the end result is also a valid flow of the same total value as the original.

Since we have shown that any flow in one digraph can be translated into a flow of the same value in the other, we can translate the maximum value flow for wone of them to get that it's max value flow is $\le$ to that of the other, and do it in the reverse direction as well to acheive equality.\\

\noindent\textbf{Exercise 26.1-3}\\
Suppose that we are in the situation posed by the question, that is, that there is some vertex $u$ that lies on no path from $s$ to $t$. Then, suppose that we have for some vertex $v$, either $f(v,u)$ or $f(u,v)$ is nonzero. Since flow must be conserved at $u$, having any positive flow either leaving or entering $u$, there is both flow leaving and entering. Since $u$ doesn't lie on a path from $s$ to $t$, we have that there are two cases, either there is no path from $s$ to $u$ or(possibly and) there is no path from $u$ to $t$. If we are in the second case, we construct a path with $c_0 = u$, and $c_{i+1}$ is an successor of $c_i$ that has $f(c_i,c_{i+1})$ being positive. Since the only vertex that is allowed to have a larger flow in than flow out is $t$, we have that this path could only ever terminate if it were to reach $t$, since each vertex in the path has some positive flow in. However, we could never reach $t$ because we are in the case that there is no path from $u$ to $t$. If we are in the former case that there is no path from $s$ to $u$, then we similarly define $c_0 = u$, however, we let $c_{i+1}$ be any vertex so that $f(c_{i+1},c_i)$ is nonzero. Again, this sequence of verticescannot terminate since we could never arrive at having $s$ as one of the vertices in the sequence.

Since in both cases, we an always keep extending the sequence of vertices, we have that it must repeat itself at some point. Once we have some cycle of vertices, we can decrease the total flow around the cycle by an amount equal to the minimum amount of flow that is going along it without changing the value of the flow from $s$ to $t$ since neither of those two vertices show up in the cycle. However, by decreasing the flow like this, we decrease the total number of edges that have a positive flow. If there is still any flow passing though $u$, we can continue to repeat this procedure, decreasing the number of edges with a postive flow by at least one. Since there are only finitely many vertices, at some point we need to have that there is no flow passing through $u$. The flow obtained after all of these steps is the desired maximum flow that the problem asks for.\\


\noindent\textbf{Exercise 26.1-5}\\
A linear programming problem consists of a set of variables, a linear function of those variables that needs to be maximized, and a a set of constraints. Our variables $x_e$ will be the amount of flow across each edge $e$. The function to maximize is $\sum_{e\hbox{ leaving }s} x_e - \sum_{e\hbox{ entering }s} x_e$. The sum of these flows is exactly equal to the value of the flow from $s$ to $t$. Now, we consider constraints. There are two types of constraints, capacity constraints and flow constraints. The capacity constraints are just $x_e \le c(e)$ where $c_e$ is the capacity of edge $e$. The flow constraints are that $\sum_{e\hbox{ leaving }v} x_e - \sum_{e\hbox{ entering }v} x_e = 0$ for all vertices $v\neq s,t$. Since this linear program captures all the same constraints, and wants to maximize the same thing, it is equivalent to the max flow problem.\\


\noindent\textbf{Exercise 26.1-7}\\
We can capture the vertex constraints by splitting out each vertex into two, where the edge between the two vertices is the vertex capacity. More formally, our new flow network will have vertices $\{0,1\} \times V$. It has an edge between $1 \times v$ and $0 \times u$ if there is an edge $(v,u)$ in the original graph, the capacity of such an edge is just $c(v,u)$. The edges of the second kind that the new flow network will have are from $0\times v$ to $1\times v$ for every $v$ with capacity $l(v)$. This new flow network will have $2|V|$ vertices and have $|V|+|E|$ edges. Lastly, we can see that this network does capture the idea that the vertices have capacities $l(v)$. This is because any flow that goes through $v$ in the original graph must go through the edge $(0\times v, 1\times v)$ in the new graph, in order to get from the edges going into $v$ to the edges going out of $v$.\\

\noindent\textbf{Exercise 26.2-1}\\
To see that equation (26.6) equals (26.7), we will show that the terms that we are throwing into the sums are all zero. That is, we will show that if $v\in V\setminus(V_1 \cup V_2)$, then $f'(s,v) = f'(v,s) = 0$. Since $v\not\in V_1$, then there is no edge from $s$ to $v$, similarly, since $v\not\in V_2$, there is no edge from $v$ to $s$. This means that there is no edge connecting $s$ and $v$ in any way. Since flow can only pass along edges, we know that there can be no flow passing directly between $s$ and $v$.\\


\noindent\textbf{Exercise 26.2-3}\\
%pictures <_< >_> <_<

\noindent\textbf{Exercise 26.2-5}\\
Since the only edges that have infinite value are those going from the supoersource or to the supersink, as long as we pick a cut that has the supersource and all the original ssources on one side, and the other side has the supersing as well as all the original sinks, then it will only cut rhough edges of finite capacity. Then, by Corollary 26.5, we have that the value of the flow is bounded above by the value of any of these types of cuts, which is finite.\\

\noindent\textbf{Exercise 26.2-7}\\
To check that $f_p$ is a flow, we make sure that it satisfies both the capacity constraints and the flow constraints. First, the capacity constraints. To see this, we recall our definition of $c_f(p)$, that is, it is the smallest residual capacity of any of the edges along the path $p$. Since we have that the residual capacity is always less than or equal to the initial capacity, we have that each value of the flow is less than the capacity. Second, we check the flow constraints, Since the only edges that are given any flow are along a path, we have that at each vertex interior to the path, the flow in from one edge is immediately cancelled by the flow out to the next vertex in the path. Lastly, we can check that its value is equal to $c_f(p)$ because, while $s$ may show up at spots later on in the path, it will be cancelled out as it leaves to go to the next vertex. So, the only net flow from $s$ is the inital edge along the path, since it (along with all the other edges) is given flow  $c_f(p)$, that is the value of the flow $f_p$.\\

\noindent\textbf{Exercise 26.2-9}\\
The augmented flow does satisfy the flow conservation property, since the sum of flow into a vertex and out of a vertex can be split into two sums each, one funing over flow in $f$ and the other running over flow in $f'$, since we have the parts are equal separately, their sums are also equal.

The capacity constraint is not satisfied by this arbitrary augmentation of flows. To see this, suppose we only have the vertices $s$ and $t$, and have a single edge from $s$ to $t$ of capacity 1. Then we could have a flow of value $1$ from $s$ to $t$, however, augmenting this flow with itself ends up putting two units along the edge from $s$ to $t$, which is greater than the capacity we can send.\\

\noindent\textbf{Exercise 26.2-11}\\
To test edge connectivity, we will take our graph as is, pick an arbitrary $s$ to be our source for the flow network, and then, we will consider every possible other selection of our sink $t$. For each of these flow networks, we will replace each (undirected) edge in the original graph with a pair of anti-symmetric edges, each of capacity 1. 

\noindent\textbf{Exercise 26.2-13}\\


\noindent\textbf{Problem 26-1}\\
\begin{enumerate}[a.]
\item
\end{enumerate}



\end{document}
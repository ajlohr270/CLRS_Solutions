\documentclass{article}
\usepackage{fancyhdr}
\usepackage{amsthm}
\usepackage{etoolbox}
\usepackage{verbatim}
\usepackage{enumerate}
\usepackage{amsmath}
\usepackage{algorithmicx}
\usepackage{algorithm}
\usepackage{algpseudocode}
\usepackage{fancybox}
\usepackage{tikz}


	
\pagestyle{fancy}
\title{Chapter 8}
\author{Michelle Bodnar, Andrew Lohr}

\newcounter{curnum}
\setcounter{curnum}{0}

\newtheorem{th1}{Exercise} 
\newcommand{\calH}{\mathcal{H}}
\newcommand{\calX}{\mathcal{X}}
\newcommand{\calA}{\mathcal{A}}
\newcommand{\calY}{\mathcal{Y}}

\begin{document}
\maketitle

\noindent\textbf{Exercise 8.1-1}\\

We can construct the graph whoose vertex set is the indices, and we place an edge between any two indices that are compared on the shortest path. We need this graph to be connected, because otherwise we could run the algorithm twice, once with everything in one component less than the other componenet, and a second time with the everything in the second component larger. As long as we mainatin the same relative ordering of the elements in each component, the algorithm will take exactly the same path, and so produce the same result. This means that there will be no difference in the output, even though there should be. For a graph on $n$ vertices, it is a well known that at least $n-1$ edges are neccesary for it to be connected, as the addition of an edge can reduce the number of connected components by at least one, and the graph with no edges has $n$ connected components.

So, it will have depth at least $n-1$.\\

\noindent\textbf{Exercise 8.1-3}\\
Suppose to a contradiction that there is a $c_1$ so that for every $n\ge k$, at least half of the inupts of length $n$ have depth at most $c_1 n$. However, there are less than $2^{c_1 n +1}$ elements in the tree of depth at most $c_1 n$. However, $1/2 n! > 1/2(n/e)^n >2^{c_1 n +1}$ so long as $n > e2^{c_1}$. This is a contradiction.

To have a $1/n$ fraction of them with small depth, similarly, we get a contradiction because $1/n n! > 2^{c_1 n+1}$ for large enough $n$.

To make an algorithm that is linear for a $1/2^n$ fraction of inputs, we yet again get a contradiction because $2^{-n} n!  > (n/2e)^n > 2^{c_1 n+1}$ for large enough $n$.

The moral of the story is that $n!$ grows very quickly.\\

\noindent\textbf{Exercise 8.2-1}\\


\noindent\textbf{Problem 1}\\

\noindent\textbf{Problem 3}\\
\noindent\textbf{Problem 5}\\

\end{document}
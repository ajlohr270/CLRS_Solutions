\documentclass{article}
\usepackage{fancyhdr}
\usepackage{amsthm}
\usepackage{etoolbox}
\usepackage{verbatim}
\usepackage{enumerate}
\usepackage{amsmath}
\usepackage{algorithmicx}
\usepackage{algorithm}
\usepackage{algpseudocode}
\usepackage{amssymb}
\usepackage{tikz}
	
\pagestyle{fancy}
\title{Appendix A}
\author{Michelle Bodnar, Andrew Lohr}

\newcounter{curnum}
\setcounter{curnum}{0}

\newtheorem{th1}{Exercise} 
\newcommand{\calH}{\mathcal{H}}
\newcommand{\calX}{\mathcal{X}}
\newcommand{\calA}{\mathcal{A}}
\newcommand{\calY}{\mathcal{Y}}

\begin{document}
\maketitle

\noindent\textbf{Exercise A.1-2}\\

Using the harmonic series formula we have that

\[\sum_{k=1}^n \frac{1}{(2k-1)} \leq 1 + \sum_{k=1}^n \frac{1}{2k} =  1 + \ln(\sqrt{n}) + O(1) =  \ln(\sqrt{n}) + O(1).\]

\noindent\textbf{Exercise A.1-4}\\

Using formula A.8 we have

\begin{align*}
\sum_{k=0}^\infty \frac{k-1}{2^k} &= -1 + \frac{1}{2} \sum_{k=0}^\infty k\left(\frac{1}{2}\right)^k \\
&= -1 + \frac{1}{2} \cdot \frac{ 1/2}{1/4} \\
&= -1 + 1 \\
&= 0.
\end{align*}

\noindent\textbf{Exercise A.1-6}\\

Let $g_1, g_2, \ldots, g_n$ be any functions such that $g_k(i) = O(f_k(i))$. By the definition of big-oh there exist constant $c_1, c_2, \ldots, c_n$ such that $g_k(i) \leq c_kf_k(i)$.  Let $c = \max_{1 \leq k \leq n} c_k$.  Then we have

 \[\sum_{k=1}^n g_k(i) \leq \sum_{k=1}^nc_kf_k(i) \leq c\sum_{k=1}^n f_k(i) = O\left(\sum_{k=1}^nf_k(i)\right).\]

\noindent\textbf{Exercise A.1-8}\\

We expand the product and cancel as follows:

\begin{align*}
\prod_{k=2}^n 1 - 1/k^2  &=  \prod_{k=2}^n \frac{(k-1)(k+1)}{k^2} \\
&= \frac{1 \cdot 3}{2 \cdot 2} \cdot \frac{2 \cdot 4}{3 \cdot 3} \cdot \frac{3 \cdot 5}{4 \cdot 4} \cdots \frac{(n-1)\cdot(n+1)}{n \cdot n} \\
&= \frac{n+1}{2n}.
\end{align*}

\noindent\textbf{Exercise A.2-2}\\

When $n = 2^m$ the sum becomes $n + n/2 + n/4 + \ldots + 1 = 2n - 1 = O(n)$.  There always exists a power of 2 which lies between $n$ and $2n$ for any choice of $n$, so let $n'$ denote the smallest power of 2 which is greater than or equal to $n$.  Then we have

\[ \sum_{k=0}^{\lfloor \lg n \rfloor} \lceil n/2^k\rceil \leq \sum_{k=0}^{\lfloor \lg n' \rfloor} \lceil n'/2^k\rceil = 2n' - 1 \leq 4n - 1 = O(n).\]


\noindent\textbf{Exercise A.2-4}\\

Since $k^3$ is monotonically increasing we use bound A.11.  For the upper bound we have

\begin{align*}
\sum_{k=1}^n k^3 &\leq \int_1^{n+1} x^3 dx \\
&= \left.\frac{x^4}{4} \right|_1^{n+1} \\
&= \frac{(n+1)^4 - 1}{4}.
\end{align*}

For the lower bound we have 

\begin{align*}
\sum_{k=1}^n k^3 &\geq \int_0^{n} x^3 dx \\
&= \left.\frac{x^4}{4} \right|_0^n \\
&= \frac{n^4}{4}.
\end{align*}
\end{document}
\documentclass{article}
\usepackage{fancyhdr}
\usepackage{amsthm}
\usepackage{etoolbox}
\usepackage{verbatim}
\usepackage{enumerate}
\usepackage{amsmath}
\usepackage{algorithmicx}
\usepackage{algorithm}
\usepackage{algpseudocode}
\input{latexdefs}

	
\pagestyle{fancy}
\rhead{Andrew Lohr}
\title{Chapter 5}
\author{Andrew Lohr}

\newcounter{curnum}
\setcounter{curnum}{0}
%\newtheorem{ex1}{Exercise}
%\newenvironment{exercise}{\pgfmathparse{\curnum +1}\def\curnum{\pgfmathresult}\newtheorem{ex\curnum}{Exercise}\label{lbl\curnum}\label{last}\begin{ex\curnum}}{\end{ex\curnum}}
%\newenvironment{exercise}{\stepcounter{curnum}
%\newtheorem{ex\value{curnum}}{Exercise}\label{lbl\value{curnum}}\begin{ex\value{curnum}}}{\end{ex\value{curnum}}}

%\newcommand{\scoretable}{
%\begin{center}\begin{tabular}{|c|c|c|}
%\hline
%Problem Number& Page Number& Score \\
%\hline
%\newcounter{myi}
%\setcounter{myi}{1}
%\loop
%\myi&\pageref{\myi} &  \\
%\stepcounter{myi}
%\ifnum \myi < \curnum
%\repeat
% \hline
%\end{tabular}
%\end{center}
%}

\newtheorem{th1}{Exercise} 
\newcommand{\calH}{\mathcal{H}}
\newcommand{\calX}{\mathcal{X}}
\newcommand{\calA}{\mathcal{A}}
\newcommand{\calY}{\mathcal{Y}}

\begin{document}
\maketitle

\begin{th1}\label{ex1}
5.1-1
\end{th1}
\begin{proof}
We may of been presented the candidates in increasing order of goodness. This would mean that we can apply transitivity to determine our preference between any two candidates
\end{proof}

\begin{th1}\label{ex2}
5.1-3
\end{th1}
\begin{proof}
\begin{algorithm}
\begin{algorithmic}[1]
\For{all eternity}
\State $a =$ BiasedRandom
\State $b =$ BiasedRandom
\If{$a>b$}
\State \Return 1
\EndIf
\If{$a<b$}
\State \Return 0
\EndIf
\EndFor
\end{algorithmic}
\end{algorithm}
Clearly since $a$ and $b$ are IID, the probability this algorithm returns one is equal to the probability it returns 0. Also, since there is a constant positive probabiltiy ($2 p(p-1)$) that the algorithm returns on each iteration of the for loop. This program will expect to go through the loop a number of times equal to:

\[
\sum_{j=0}^{\infty} j (1 - 2 p(p-1))^j ( 2 p (p-1)) = \frac{2p(p-1) (1 - 2p(p-1))}{( 2p(p-1))^2} = \frac{1 - 2p(p-1)}{2p(p-1)}
\]
Note that the formula used for the sum of $j\alpha^j$ can be obtained by differentiawting both sides of the geometric sum formula for $\alpha^j$ with respect to $\alpha$

\end{proof}

\begin{th1}\label{ex3}
5.2-1
\end{th1}
\begin{proof}
You will hire exactly one time if the best candidate is presented first. There are $(n-1)!$ orderings with the best candidate first, so, it is with probability $\frac{(n-1)!}{n!} = \frac{1}{n}$ that you only hire once.
You will hire exactly $n$ times if the candidates are presented in increasing order. This fixes the ordering to a single one, and so this will occur with probability $\frac{1}{n!}$.
\end{proof}
\begin{th1}\label{ex4}
5.2-3
\end{th1}
\begin{proof}
Let $X_j$ be the indicator of a dice coming up $j$. So, the expected value of a single dice roll $X$ is 
\[
E[X] = \sum_{j=1}^{6} j \Pr(X_j) = \frac{1}{6} \sum_{j=1}^6 j 
\]

So, the sum of $n$ dice has probability

\[
E[nX] = nE[X] = \frac{n}{6} \sum_{j=1}^6 j = \frac{n 6(6+1)}{12} = 3.5 n
\]


\end{proof}

\begin{th1}\label{ex5}
5.2-5
\end{th1}
\begin{proof}
Let $X_{i,j}$ for $i<j$ be the indicator of $A[i] > A[j]$. Then, we have that the expected number of inversions is
\[
E\left[\sum_{i<j} X_{i,j}\right] = \sum_{i<j} E[X_{i,j}] = \sum_{i=1}^{n-1} \sum_{j=i+1}^n \Pr(A[i] > A[j]) = \frac{1}{2} \sum_{i=1}^{n-1} n-i \]\[= \frac{n(n-1)}{2} - \frac{n(n-1)}{4} = \frac{n(n-1)}{4}  
\]

\end{proof}
\begin{th1}\label{ex6}
5.3-1
\end{th1}
\begin{proof}
We modify the algorithm by unrolling the $i=1$ case.
\begin{algorithm}
\begin{algorithmic}[1]
\State swap $A[1]$ with $A[$Random(1,n)$]$
\For{i from 2 to n}
\State swap $A[i]$ with $A[$Random(i,n)$]$
\EndFor
\end{algorithmic}
\end{algorithm}


Modify the proof of the lemma by starting with $i=2$ instead of $i=1$. This entirely sidesteps the issue of talking about $0$- permutations.
\end{proof}
\begin{th1}\label{ex7}
5.3-3
\end{th1}
\begin{proof}
Consider the case of $n=3$ in running the algorithm, three IID choices will be made, and so you'll end up having 27 possible end states each with equal probability. There are $3!=6$ possible orderings, these shuld appear equally often, but this can't happen because 6 does not divide 27
\end{proof}
\begin{th1}\label{ex8}
5.3-5
\end{th1}
\begin{proof}
Let $X_{i,j}$ be the event that $P[i] = P[j]$. Then, the event that all are unique is the compliment of there being some pair that are equal, so, we must show that $\Pr(\cup_{i,j} X_{i,j}) \le 1/n$. We start by applying a union bound
\[
\Pr(\cup_{i<k} X_{i,j}) \le \sum_{i=1}^{n-1} \sum_{j=i+1}^n \Pr(X_{i,j}) = \sum_{i=1}^{n-1} \sum_{j=i+1}^n \frac{1}{n^3}
\]
Where we use the fact that any two indices will be equal with probability equal to one over the size of the probability space being drawn from, which is $n^3$.
\[
 = \sum_{i=1}^{n-1} \frac{n-i}{n^3} = \frac{n(n-1)}{n^3} - \frac{n(n-1)}{2n^3} = \frac{n-1}{2n^2} < \frac{1}{n}
\]

\end{proof}
\begin{th1}\label{ex9}
5.3-7
\end{th1}
\begin{proof}
We prove that it produces a rawndom $m$ subset by induction on $m$. It is obviously true if $m=0$ as there is only one size $m$ subset of $[n]$. Suppose $S$ is a uniform $m-1$ subset of $n-1$, that is, $\forall j\in[n-1], \Pr[j\in S] = \frac{m-1}{n-1}$. Then, if we let $S'$ denote the returned set, suppose first $j\in [n-1]$, $\Pr[j\in S'] = \Pr[j\in S] + \Pr[j\not\in S \wedge  i=j] = \frac{m-1}{n-1} + \Pr[j\not\in S]\Pr[i=j] = \frac{m-1}{n-1} + \left(1 - \frac{m-1}{n-1}\right)\frac{1}{n} = \frac{n(m-1) +n-m}{(n-1)n} = \frac{nm -m}{(n-1)n} = \frac{m}{n}$. Since the constructed subset contains each of $[n-1]$ with the correct probability, it must also contain $n$ with the correct probability because the probabilities sum to 1.
\end{proof}
\begin{th1}\label{ex10}
5.4-1
\end{th1}
The probability that none of $n$ people have the same birthday as you is $(1-\frac{1}{365})^n= \frac{364^n}{365^n}$. This falls below zero when $n\ge \log_{\frac{364}{365}}(.5) \approx 252.6$ so, when $n=253$. Since you are also a person in the room, we add one to get the final anwer of $254$.

The probability that k of the n people have july 4 as a birthday is $\binom{n}{k} \frac{364^{n-k}}{365}$. In particular, for $k=0$ it is $\frac{364^n}{365^n}$ and for $k=1$ it is $\frac{n364^{n-1}}{365^n}$. Adding these up and solving for $n$ to have the sum drop less than a half, we get
\[
\left(\frac{364}{365}\right)^n \left( 1 + \frac{n}{364}\right) <.5
\]
This is difficult to solve analytically, but  because of the LHS's monotonicity, the answer can be found rather quickly by using a gallop search to be $n=612$.
\begin{proof}

\end{proof}
\begin{th1}\label{ex11}
5.4-3
\end{th1}
\begin{proof}
Pairwise independence is sufficient. All the independence is used for is to show that $\Pr(b_i=r \wedge b_j=r) = \Pr(b_i=r)\Pr(b_j=r)$. This is a result of pairwise independence.
\end{proof}
\begin{th1}\label{ex12}
5.4-5
\end{th1}
\begin{proof}
Since to be a k-permutation, we need that no letter appears repeated, it is equivalent to the birthday problem with k people and n days. So, this probability is given on the top of page 132 to be:
\[
\prod_{i=1}^{k-1} \left(1-\frac{i}{n}\right)
\]
\end{proof}
\begin{th1}\label{ex13}
5.4-7
\end{th1}
\begin{proof}
We split up the $n$ flips into $n/s$ groups where we pick $s = \lg(n) - 2\lg(\lg(n))$. We will show that at least one of these groups comes up all heads with probability at least $\frac{n-1}{n}$. So, the probability the group starting in position $i$ comes up all heads is:
\[
\Pr(A_{i,\lg(n)- 2\lg(\lg(n))}) =\frac{1}{2^{\lg(n)- 2\lg(\lg(n))}} = \frac{\lg(n)^2}{n}
\]
Since the groups are based of of disjoint sets of IID coinflips, these probabilities are indipendent. so, 
\[
\Pr(\bigwedge_{i} \neg A_{i,\lg(n)- 2\lg(\lg(n))}) = \prod_i \Pr(\neg A_{i,\lg(n)- 2\lg(\lg(n))}) \]\[= \left(1 - \frac{\lg(n)^2}{n}\right)^{\frac{n}{\lg(n) - 2\lg(\lg(n))}} \le e^{-\frac{\lg(n)^2}{\lg(n) - 2\lg(\lg(n))}} = \frac{1}{n}e^{\frac{-2\lg(\lg(n))\lg(n)}{\lg(n) - 2\lg(\lg(n))}} 
\]
\[
= n^{-1 - \frac{2\lg(\lg(n))}{\lg(n) - 2\lg(\lg(n))}} < n^{-1}
\]
Showing that the probability that there is no run of length at least $\lg(n) - 2\lg(\lg(n))$ to be $<\frac{1}{n}$.

\end{proof}
\begin{th1}\label{ex14}
5-1
\end{th1}
\begin{proof}
\begin{enumerate}[a)]
\item
Let $P_{j,i}$ be the probability that the counter reads $j$ after $i$ steps. The rule for incrementing the counter says exactly that
\[
P_{j,i} = \frac{P_{j-1,i-1}}{n_{i} - n_{i-1}} + \left( 1 - \frac{1}{n_{i+1} - n_i}\right)P_{j-1,i} 
\]
And we are to show that the expected value after $j$ steps is $j$, that is,
\[
\sum_{i=0}
\]

\item

\end{enumerate}
\end{proof}



\end{document}
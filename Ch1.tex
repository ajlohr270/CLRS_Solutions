\documentclass{article}
\usepackage{fancyhdr}
\usepackage{amsthm}
\usepackage{etoolbox}
\usepackage{verbatim}
\usepackage{enumerate}
\usepackage{amsmath}
\input{latexdefs}

	
\pagestyle{fancy}
\rhead{Andrew Lohr}
\title{Chapter 1}
\author{Andrew Lohr}

\newcounter{curnum}
\setcounter{curnum}{0}
%\newtheorem{ex1}{Exercise}
%\newenvironment{exercise}{\pgfmathparse{\curnum +1}\def\curnum{\pgfmathresult}\newtheorem{ex\curnum}{Exercise}\label{lbl\curnum}\label{last}\begin{ex\curnum}}{\end{ex\curnum}}
%\newenvironment{exercise}{\stepcounter{curnum}
%\newtheorem{ex\value{curnum}}{Exercise}\label{lbl\value{curnum}}\begin{ex\value{curnum}}}{\end{ex\value{curnum}}}

%\newcommand{\scoretable}{
%\begin{center}\begin{tabular}{|c|c|c|}
%\hline
%Problem Number& Page Number& Score \\
%\hline
%\newcounter{myi}
%\setcounter{myi}{1}
%\loop
%\myi&\pageref{\myi} &  \\
%\stepcounter{myi}
%\ifnum \myi < \curnum
%\repeat
% \hline
%\end{tabular}
%\end{center}
%}

\newtheorem{th1}{Exercise} 
\newcommand{\calH}{\mathcal{H}}
\newcommand{\calX}{\mathcal{X}}
\newcommand{\calA}{\mathcal{A}}
\newcommand{\calY}{\mathcal{Y}}

\begin{document}
\maketitle

\begin{th1}\label{ex1}
1.1-1
\end{th1}
\begin{proof}
An example of a real world situation that would require sorting would be if you wanted to keep track of a bunch of people's file folders and be able to look up a given name quickly. A convex hull might be needed if you needed to secure a wildlife santuary with fencing and had to contain a bunch of specific nesting locations.
\end{proof}

\begin{th1}\label{ex2}
1.1-3
\end{th1}
\begin{proof}
An array. It has the limitation of requiring a lot of copying when resizing, inserting, and removing elements. 
\end{proof}

\begin{th1}\label{ex3}
1.1-5
\end{th1}
\begin{proof}
If you were for example keeping track of terror watch suspects, it would be unacceptable to have it occasionally bringing up a wrong decision as to whether a person is on the list or not. It would be fine to only have an approximate solution to the shortest route on which to drive, an extra little bit of driving is not that bad.
\end{proof}
\begin{th1}\label{ex4}
1.2-1
\end{th1}
\begin{proof}
A program that would pick out which music a user would like to listen to next. They would need to use a bunch of information from historical and popular preferences in order to maximize.
\end{proof}

\begin{th1}\label{ex5}
1.2-3
\end{th1}
\begin{proof}
We want that $100n^2<2^n$. note that if $n=14$, this becomes $100(14)^2=19600 > 2^14  = 16384$. For $n=15$ it is $100(15)^2 = 22500 < 2^{15} = 32768$. So, the answer is $n=15$.

\end{proof}
\begin{th1}\label{ex6}
1-1
\end{th1}
\begin{proof}
We assume a 30 dday month and 365 day year.

$
\begin{array}{c|c|c|c|c|c|c|c|}
&1 Second&1 Minute &1 Hour&1 Day&1 Month&1 Year&1 Century\\
\hline
\lg n& 2^{1\times10^6}&2^{6\times10^7} &2^{3.6\times10^9} &2^{8.64\times10^{10}} &2^{2.592\times10^{12}} & 2^{3.1536\times10^{13} } & 2^{3.15576\times10^{15}} \\
\hline
\sqrt{n} &1\times10^{12} &3.6\times10^{15} &1.29\times10^{19} & 7.46\times10^{21}& 6.72\times10^{24}& 9.95\times10^{26}&9.96\times10^{30} \\
\hline
n &1\times10^6&6\times10^7 &3.6\times10^9 &8.64\times10^{10} &2.59\times10^{12}  & 3.15\times10^{13} & 3.16\times10^{15}\\
\hline
n \lg n & 189481& 8.64\times10^6&4.18\times10^8 &8.69\times10^9 &2.28\times10^{11} & 2.54\times10^{12}&2.20\times10^{14} \\
\hline
n^2 &1000 & 7745& 60000&293938 & 1609968& 5615692& 56176151\\
\hline
n^3 & 100& 391& 1532& 4420& 13736&31593 &146679 \\
\hline
2^n & 19 & 25& 31&36 &41 & 44 &51 \\
\hline
n! & 9&11 & 12& 13& 15& 16& 17\\
\hline
\end{array}
$
\end{proof}
\end{document}
\documentclass{article}
\usepackage{fancyhdr}
\usepackage{amsthm}
\usepackage{etoolbox}
\usepackage{verbatim}
\usepackage{enumerate}
\usepackage{amsmath}
\usepackage{algorithmicx}
\usepackage{algorithm}
\usepackage{algpseudocode}
\usepackage{amssymb}
\usepackage{tikz}
	
\pagestyle{fancy}
\title{Chapter 26}
\author{Michelle Bodnar, Andrew Lohr}

\newcounter{curnum}
\setcounter{curnum}{0}

\newtheorem{th1}{Exercise} 
\newcommand{\calH}{\mathcal{H}}
\newcommand{\calX}{\mathcal{X}}
\newcommand{\calA}{\mathcal{A}}
\newcommand{\calY}{\mathcal{Y}}

\begin{document}
\maketitle
\noindent\textbf{Exercise A.1-1}\\

\[
\sum_{k=1}^{n} (2k -1) = 2\sum_{k=1}^{n} k - \sum_{k=1}^n 1 = n(n+1) - n = n^2
\]

\noindent\textbf{Exercise A.1-3}\\
First, we recall equation (A.8)
\[
\sum_{k=0}^\infty k x^k = \frac{x}{(1-x)^2}
\]
for $|x|<1$. Then, we take a derivative of each side, taking the derivative of the left hand side term by term
\[
\sum_{k=0}^\infty k\cdot kx^{k-1} = \frac{(1-x)^2 + 2x (1-x)}{(1-x)^4} = \frac{(1-x) + 2x}{(1-x)^3} = \frac{(1+x)}{(1-x)^3}
\]
Lastly, since we have a $x^{k-1}$ instead of the $x^k$ that we'd like, we'll multiply both sides of the equation by $x$ to get the desired equality.
\[
\sum_{k=0}^\infty k^2 x^k = \frac{x(1+x)}{(1-x)^3}
\]

\noindent\textbf{Exercise A.1-5}\\

First, we'll start with the equation 
\[
\sum_{k=0}^\infty y^k = \frac{1}{1-y}
\]
So long as $|y|<1$. Then, we'll let $y = x^2$ to get 
\begin{align*}
\sum_{k=0}^\infty (x^2)^k &= \frac{1}{1-x^2}\\
\sum_{k=0}^\infty xx^{2k} &= \frac{x}{1-x^2}\\
\sum_{k=0}^\infty x^{2k+1} &= \frac{x}{1-x^2}\\
\sum_{k=0}^\infty (2k+1) x^{2k} &= \frac{(1-x^2) +2 x^2 }{(1-x^2)^2}\\
\sum_{k=0}^\infty (2k+1) x^{2k} &= \frac{1 +x^2 }{(1-x^2)^2}
\end{align*}
so long as $|x|<1$.\\

\noindent\textbf{Exercise A.1-7}\\

\begin{align*}
&\lg\left(\prod_{k=1}^n 2\cdot 4^k\right)\\
&=\sum_{k=1}^n \lg(2 \cdot 4^k) \\
&=\sum_{k=1}^n \lg(2) + k \lg(4) \\
&=\left(\lg(2) \sum_{k=1}^n 1\right) + \left(\lg(4) \sum_{k=1}^n k\right) \\
&=n + 2 \frac{n(n+1)}{2}\\
&= n(n+2)
\end{align*}

This means that we need to raise 2 to this quantity to get the desired product, so out final answer is
\[
2^{n(n+2)} = 2^{n^2} \cdot 4^n
\]

\noindent\textbf{Exercise A.2-1}\\
Define a function $f_1 = \lceil \frac{1}{x^2} \rceil$ and $f_2 = 1+ \frac{1}{x^2}$. Note that we always have that $f_1 \le f_2$. Then we have that the desired summation is exactly equal to $\int_1^\infty f_1$ because the graph of $f_1$ is a bunch of rectangles of width 1 and height equal to each of the terms in the sum. By monotonicity of integrals, we have that this is $ \le \int_1^\infty f_2 =2$. \\

\noindent\textbf{Exercise A.2-3}\\


\noindent\textbf{Exercise A.2-5}\\

\noindent\textbf{Problem A-1}\\
\begin{enumerate}[a.]
\item

\end{enumerate}


\end{document}
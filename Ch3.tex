\documentclass{article}
\usepackage{fancyhdr}
\usepackage{amsthm}
\usepackage{etoolbox}
\usepackage{verbatim}
\usepackage{enumerate}
\usepackage{amsmath}
\usepackage{algorithmicx}
\usepackage{algorithm}
\usepackage{algpseudocode}
\usepackage{amssymb}
	
\pagestyle{fancy}
\title{Chapter 3}
\author{Michelle Bodnar, Andrew Lohr}

\newcounter{curnum}
\setcounter{curnum}{0}

\newtheorem{th1}{Exercise} 
\newcommand{\calH}{\mathcal{H}}
\newcommand{\calX}{\mathcal{X}}
\newcommand{\calA}{\mathcal{A}}
\newcommand{\calY}{\mathcal{Y}}

\begin{document}
\maketitle

\noindent\textbf{Exercise 3.1-1}\\

Since we are requiring both $f$ and $g$ to be aymptotically non-negative, suppose that we are past some $n_1$ where both are non-negative (take the max of the two bounds on the n corresponding to both $f$ and $g$). Let $c_1=.5$ and $c_2=1$.
\[
0\le .5(f(n)+g(n)) \le .5(\max(f(n),g(n)) + \max(f(n),g(n))) \]\[= \max(f(n),g(n)) \le \max(f(n),g(n))+\min(f(n),g(n)) = (f(n)+g(n))
\]\\

\noindent\textbf{Exercise 3.1-2}\\

Let $c = 2^b$ and $n_0 \geq 2a$.  Then for all $n \geq n_0$ we have $(n+a)^b \leq (2n)^b = cn^b$ so $(n+a)^b = O(n^b)$.  Now let $n_0 \geq \frac{-a}{1-1/2^{1/b}}$ and $c = \frac{1}{2}$.  Then $n \geq n_0 \geq \frac{-a}{1-1/2^{1/b}}$ if and only if $n - \frac{n}{2^{1/b}} \geq -a$ if and only if $n+a \geq (1/2)^{a/b}n$ if and only if $(n+a)^b \geq cn^b$.  Therefore $(n+a)^b = \Omega(n^b)$.  By Theorem 3.1, $(n+a)^b = \Theta(n^b)$.\\

\noindent\textbf{Exercise 3.1-3}\\

There are a ton of different funtions that have growth rate less than or equal to $n^2$. In particular, functions that are constant or shrink to zero arbitrarily fast. Saying that you grow more quickly than a function that shrinks to zero quickly means nothing. \\

\noindent\textbf{Exercise 3.1-4}\\

$2^{n+1} \geq 2\cdot 2^n$ for all $n \geq 0$, so $2^{n+1} = O(2^n)$.  However, $2^{2n}$ is not $O(2^n)$.  If it were, there would exist $n_0$ and $c$ such that $n \geq n_0$ implies $2^n \cdot 2^n = 2^{2n} \leq c2^n$, so $2^n \leq c$ for $n \geq n_0$ which is clearly impossible since $c$ is a constant.  \\


\noindent\textbf{Exercise 3.1-5}\\

Suppose $f(n)\in \Theta(g(n))$, then $\exists c_1,c_2,n_0, \forall n\ge n_0, 0\le c_1 g(n) \le f(n) \le c_2 g(n)$, if we just look at these inequalities saparately, we have $c_1 g(n) \le f(n)$ ($f(n) \in \Omega(g(n))$) and $f(n) \le c_2 g(n)$ ($f(n)\in O(g(n))$).

Suppose that we had $\exists n_1, c_1, \forall n\ge n_1, c_1 g(n) \le f(n)$ and $\exists n_2,c_2, \forall n\ge n_2, f(n)\le c_2g(n)$. Putting these together, and letting $n_0 = \max(n_1,n_2)$, we have $\forall n\ge n_0, c_1 g(n) \le f(n) \le c_2 g(n)$. \\

\noindent\textbf{Exercise 3.1-6}\\

Suppose the running time is $\Theta(g(n))$.  By Theorem 3.1, the running time is $O(g(n))$, which implies that for any input of size $n \geq n_0$ the running time is bounded above by $c_1g(n)$ for some $c_1$. This includes the running time on the worst-case input.  Theorem 3.1 also imlpies the running time is $\Omega(g(n))$, which implies that for any input of size $n \geq n_0$ the running time is bounded below by $c_2g(n)$ for some $c_2$.  This includes the running time of the best-case input.  

On the other hand, the running time of any input is bounded above by the worst-case running time and bounded below by the best-case running time.  If the worst-case and best-case running times are $O(g(n))$ and $\Omega(g(n))$ respectively, then the running time of any input of size $n$ must be $O(g(n))$ and $\Omega(g(n))$.  Theorem 3.1 implies that the running time is $\Theta(g(n))$. \\

\noindent\textbf{Exercise 3.1-7}\\

Suppose we had some $f(n) \in o(g(n)) \cap \omega(g(n))$. Then, we have
\[
0 = \lim_{n\rightarrow \infty} \frac{f(n)}{g(n)} = \infty
\]
a contradiction.\\

\noindent\textbf{Exercise 3.1-8}\\

\begin{align*}
 \Omega(g(n,m)) = \{ f(n,m) &: \mbox{there exist positive constants } c, n_0, \mbox{ and }m_0 \mbox{ such that } f(n,m) \geq cg(n,m) \\
&\mbox{ for all } n \geq n_0 \mbox{ or } m \geq m_0 \}
\end{align*}

\begin{align*}
\Theta(g(n,m)) = \{ f(n,m) &: \mbox{there exist positive constants } c_1, c_2, n_0, \mbox{ and }m_0 \mbox{ such that } c_1g(n,m) \leq f(n,m) \\
&\leq c_2g(n,m) \mbox{ for all } n \geq n_0 \mbox{ or } m \geq m_0 \}
\end{align*}

\noindent\textbf{Exercise 3.2-1}\\

Let $n_1 < n_2$ be arbitrary. From $f$ and $g$ being monatonic increasing, we know $f(n_1)< f(n_2)$ and $g(n_1) < g(n_2)$. So
\[
f(n_1)+g(n_1) < f(n_2) + g(n_1) < f(n_2)+g(n_2)
\]
Since $g(n_1)<g(n_2)$, we have $f(g(n_1))<f(g(n_2))$. Lastly, if both are nonegative, then, 
\[
f(n_1)g(n_1) = f(n_2)g(n_1) + (f(n_2)-f(n_1))g(n_1) \]\[= f(n_2)g(n_2) + f(n_2)(g(n_2)-g(n_1)) +(f(n_2)-f(n_1))g(n_1) 
\]
Since $f(n_1)\ge 0$, $f(n_2)>0$, so, the second term in this expression is greater than zero. The third term is nonnegative, so, the whole thing is$<f(n_2)g(n_2)$.\\

\noindent\textbf{Exercise 3.2-2}\\

\[ a^{\log_b(c)} = a^{\frac{\log_a(c)}{\log_a(b)}} = c^{\frac{1}{\log_a(b)}} = c^{\log_b(a)}.\]

\noindent\textbf{Exercise 3.2-3}\\

As the hint suggests, we will apply stirling's approximation

\[
\lg(n!) = \lg\left(\sqrt{(2\pi n}\left(\frac{n}{e}\right)^n\left( 1 + \Theta\left(\frac{1}{n}\right)\right)\right) \]\[= \frac{1}{2}\lg(2\pi n) + n\lg(n) - n \lg (e) + \lg\left(\Theta\left(\frac{n+1}{n}\right)\right)
\]
Note that this last term is $O(\lg(n))$ if we just add the two expression we get when we break up the $\lg$ instead of subtract them. So, the whole expression is dominated by $n\lg(n)$. So, we have that $\lg(n!) = \Theta(n\lg(n))$.

\[
\lim_{n\rightarrow\infty} \frac{2^n}{n!} = \lim_{n\rightarrow\infty} \frac{1}{\sqrt{2\pi n}(1+\Theta(\frac{1}{n}))} \left(\frac{2e}{n}\right)^n \le  \lim_{n\rightarrow\infty} \left(\frac{2e}{n}\right)^n
\]
If we restrict to $n> 4e$, then this is 
\[
\le \lim_{n\rightarrow\infty} \frac{1}{2^n} = 0
\]

\[
\lim_{n\rightarrow\infty} \frac{n^n}{n!} = \lim_{n\rightarrow\infty} \frac{1}{\sqrt{2\pi n}(1+\Theta(\frac{1}{n}))}e^n =  \lim_{n\rightarrow\infty} O(n^{-.5})e^n \ge \lim_{n\rightarrow\infty} \frac{e^n}{c_1\sqrt{n}}\]\[ \ge\lim_{n\rightarrow\infty} \frac{e^n}{c_1n} =  \lim_{n\rightarrow\infty} \frac{e^n}{c_1} = \infty
\]\\

\noindent\textbf{Exercise 3.2-4}\\

The function $\lceil \log n \rceil !$ is not polynomially bounded.  If it were, there would exist constants $c$, $a$, and $n_0$ such that for all $n \geq n_0$ the inequality $\lceil \log n \rceil ! \leq cn^a$ would hold.  In particular, it would hold when $n=2^k$ for $k \in \mathbb{N}$.  Then this becomes $k! \leq c(2^a)^k$, a contradiction since the factorial function is not exponentially bounded.

We'll show that $\lceil \log \log n \rceil ! \leq n$.  Without loss of generality assume $n=2^{2^k}$.  Then this becomes equivalent to showing $k! \leq 2^{2^k}$, or $1 \cdot 2 \cdots (k-1) \cdot k \leq 4 \cdot 16 \cdot 2^8 \cdots 2^{2^k}$, which is clearly true for $k \geq 1$.  Therefore it is polynomially bounded.



\noindent\textbf{Exercise 3.2-5}\\

Note that $\lg^*(2^n) = 1+ \lg^*(n)$, so, 
\begin{align*}
\lim_{n\rightarrow \infty} \frac{\lg(\lg^*(n))}{\lg^*(\lg(n))}&= \lim_{n\rightarrow \infty} \frac{\lg(\lg^*(2^n))}{\lg^*(\lg(2^n))}\\
& = \lim_{n\rightarrow \infty} \frac{\lg(1+\lg^*(n))}{\lg^*(n)} \\
&= \lim_{n\rightarrow \infty} \frac{\lg(1+n)}{n} \\
&= \lim_{n\rightarrow \infty} \frac{1}{1+n} \\
&= 0
\end{align*}

So, we have that $\lg^*(\lg(n))$ grows more quickly\\

\noindent\textbf{Exercise 3.2-6}\\

\[ \phi^2 = \left(\frac{1 + \sqrt{5}}{2} \right)^2 = \frac{6 + 2\sqrt{5}}{4} = 1 + \frac{1 + \sqrt{5}}{2} = 1 + \phi\]

\[ \hat{\phi}^2 = \left(\frac{1 - \sqrt{5}}{2} \right)^2 = \frac{6 - 2\sqrt{5}}{4} = 1 + \frac{ 1- \sqrt{5}}{2} = 1 + \hat{\phi} \]

\noindent\textbf{Exercise 3.2-7}\\

First, we show that $1+\phi = \frac{6+2\sqrt{5}}{4} = \phi^2$. So, for every $i$, $\phi^{i-1}+ \phi^{i-2} = \phi^{i-2}(\phi+1) = \phi^i$. Similarly for $\hat\phi$.

For $i=0$, $\frac{\phi^0 - \hat\phi^0}{\sqrt{5}} = 0$. For $i=1$, $\frac{\frac{1+\sqrt{5}}{2} - \frac{1-\sqrt{5}}{2}}{\sqrt{5}} = \frac{\sqrt{5}}{\sqrt{5}} = 1$. Then, by induction, $F_i = F_{i-1}+F_{i-2} = \frac{\phi^{i-1}+\phi^{i-2} - (\hat\phi^{i-1} +\hat\phi^{i-2})}{\sqrt{5}} = \frac{\phi^i - \hat\phi^i}{\sqrt{5}}$.\\

\noindent\textbf{Exercise 3.2-8}\\

Let $c_1$ and $c_2$ be such that $ c_1 n \leq k \ln k \leq c_2 n$.  Then we have $ \ln c_1 + \ln n= \ln(c_1 n) \leq \ln(k \ln k) = \ln k + \ln ( \ln k)$ so $\ln n = O(\ln k)$.  Let $c_3$ be such that $\ln n \leq c_3 \ln k$.  Then 
\[ \frac{n}{\ln n} \geq \frac{n}{c_3 \ln k} \geq \frac{k}{c_2 c_3} \]
so that $\frac{n}{\ln n} = \Omega(k)$.  Similarly, we have $ \ln k + \ln ( \ln k) = \ln (k \ln k) \leq \ln(c_2 n) = \ln(c_2) + \ln(n)$ so $\ln(n) = \Omega(\ln k)$.  Let $c_4$ be such that $\ln n \geq c_4 \ln k$.  Then
\[ \frac{n}{\ln n} \leq \frac{n}{c_4 \ln k} \leq \frac{k}{c_1 c_4} \]
so that $\frac{n}{\ln n} = O(k)$.  By Theorem 3.1 this implies $\frac{n}{\ln n} = \Theta(k)$.  By symmetry, $k = \Theta\left(\frac{n}{\ln n}\right)$. \\

\noindent\textbf{Problem 3-1}\\

\begin{enumerate}[a.]
\item
If we pick any $c>0$, then, the end behavior of $cn^k -p(n)$ is going to infinity, in particular, there is an $n_0$ so that for every $n\ge n_0$, it is positive, so, we can add $p(n)$ to both sides to get $p(n)<cn^k$.

\item
If we pick any $c>0$, then, the end behavior of $p(n)- cn^k$ is going to infinity, in particular, there is an $n_0$ so that for every $n\ge n_0$, it is positive, so, we can add $cn^k$ to both sides to get $p(n)>cn^k$.

\item
We have by the previous parts that $p(n) = O(n^k)$ and $p(n)= \Omega(n^k)$. So, by Theorem 3.1, we have that $p(n) = \Theta(n^k)$.

\item
\[
\lim_{n\rightarrow\infty} \frac{p(n)}{n^k} = \lim_{n\rightarrow\infty} \frac{n^d(a_d + o(1))}{n^k} < \lim_{n\rightarrow\infty} \frac{2a_dn^d}{n^k} = 2a_d\lim_{n\rightarrow\infty} n^{d-k} = 0
\]


\item
\[
\lim_{n\rightarrow\infty} \frac{n^k}{p(n)} = \lim_{n\rightarrow\infty} \frac{n^k}{n^d O(1)} < \lim_{n\rightarrow\infty} \frac{n^k}{n^d} = \lim_{n\rightarrow\infty} n^{k-d} = 0
\]

\end{enumerate}

\noindent\textbf{Problem 3-2}\\

\begin{tabular}{c|c|c|c|c|c|c|}
$A$ & $B$ & $O$ & $o$ & $\Omega$ & $\omega$ & $\Theta$ \\ \hline
$\lg^k n$ & $n^\epsilon$ & yes & yes & no & no & no \\ \hline
$n^k$ & $c^n$ & yes & yes & no & no & no \\ \hline 
$\sqrt{n}$ & $n^{\sin n}$ & no & no & no & no & no \\ \hline
$2^n$ & $2^{n/2}$ & no & no & yes & yes & no \\ \hline
$n^{\log c}$ & $c^{\log n}$ & yes & no & yes & no & yes \\ \hline
$\log(n!)$ & $\log(n^n)$ & yes & no & yes & no & yes \\ \hline
\end{tabular} \\

\noindent\textbf{Problem 3-3}\\
\begin{enumerate}[a.]
\item
$
\begin{array}{|c c|}
2^{2^{n+1}}&\\
2^{2^n}&\\
(n+1)!&\\
n!&\\
n2^n&\\
e^n&\\
2^n&\\
\left(\frac{3}{2}\right)^n&\\
(\lg(n))!&\\
 n^{\lg(\lg(n))}&\lg(n)^{\lg(n)}\\
 n^3&\\
  n^2& 4^{\lg(n)}\\
  n\lg(n)&\lg(n!)\\
  2^{\lg(n)}&n\\
  (\sqrt{2})^{\lg(n)}&\\
  2^{\sqrt{2\lg(n)}}&\\
  \lg^2(n)&\\
  \lg(n)&\\
  \sqrt{\lg(n)}&\\
  \ln(\ln(n))&\\
  2^{\lg^*(n)}&\\
  \lg^*(n) &\lg^*(\lg(n))\\
  \lg(\lg^*(n)&\\
  1&n^{1/\lg(n)}

\end{array}
$


The terms are in decreasing growth rate by row. Functions in the same row are $\Theta$ of each other.
\item

\[
f(n) = \left\{\begin{array}{c c}g_1(n)!& n\mod 2=0 \\0 &n\mod 2 =1\end{array}\right.
\]
\end{enumerate}

\noindent\textbf{Problem 3-4}\\
\begin{enumerate}[a.]
\item False.  Counterexample: $n = O(n^2)$ but $n^2 \neq O(n)$. \\

\item False.  Counterexample: $n + n^2 \neq \Theta(n)$. \\

\item True.  Since $f(n) = O(g(n))$ there exist $c$ and $n_0$ such that $n \geq n_0$ implies $f(n) \leq cg(n)$ and $f(n) \geq 1$.  This means that $\log(f(n)) \leq \log(cg(n)) = \log(c) + \log(g(n))$.  Note that the inequality is preserved after taking logs because $f(n) \geq 1$.  Now we need to find $d$ such that $f(n) \leq d\log(g(n))$.  It will suffice to make $\log(c) + \log(g(n)) \leq d\log(g(n))$, which is achieved by taking $d = \log(c) + 1$, since $\log(g(n)) \geq 1$.\\

\item False.  Counterexample: $2n = O(n)$ but $2^{2n} \neq 2^{n}$ as shown in exercise 3.1-4. \\

\item False.  Counterexample: Let $f(n) = \frac{1}{n}$. Suppose that $c$ is such that $\frac{1}{n} \leq c \frac{1}{n^2}$ for $n \geq n_0$.  Choose $k$ such that $kc \geq n_0$ and $k > 1$.  Then this implies $\frac{1}{kc} \leq \frac{c}{k^2c^2} = \frac{1}{k^2 c}$, a contradiction.  \\

\item True.  Since $f(n) = O(g(n))$ there exist $c$ and $n_0$ such that $n \geq n_0$ implies $f(n) \leq c g(n)$.  Thus $g(n) \geq \frac{1}{c} f(n)$, so $g(n) = \Omega(f(n))$. \\

\item False.  Counterexample: Let $f(n) = 2^{2n}$.  By exercise 3.1-4, $2^{2n} \neq O(2^n)$. \\

\item True.  Let $g$ be any function such that $g(n) = o(f(n))$.  Since $g$ is asymptotically positive let $n_0$ be such that $n \geq n_0$ implies $g(n) \geq 0$.  Then $f(n) + g(n) \geq f(n)$ so $f(n) + o(f(n)) = \Omega(f(n))$.  Next, choose $n_1$ such that $n \geq n_1$ implies $g(n) \leq f(n)$.  Then $f(n) + g(n) \leq f(n) + f(n) = 2f(n)$ so $f(n) + o(f(n)) = O(f(n))$.  By Theorem 3.1, this implies $f(n) + o(f(n)) = \Theta(f(n))$. 
\end{enumerate}
\noindent\textbf{Problem 3-5}\\

\begin{enumerate}[a.]
\item
Suppose that we do not have that $f=O(g(n))$. This means that $\forall c>0,n_0, \exists n\ge n_0, f(n) > c g(n)$. Since this holds for every $c$, we can let it be arbitrary, say 1. Initially, we set $n_0=1$, then, the resulting $n$ we will call $a_1$. Then, in general, let $n_0= a_i+1$ and let $a_{i+1}$ be the resulting value of $n$. Then, on the infinite set $\{a_1,a_2,\ldots\}$, we have $f(n)>g(n)$, and so, $f = \overset{\infty}{\Omega}(g(n))$

This is not the case for the usual definition of $\Omega$. Suppose we had $f(n) = n^2(n\mod 2)$ and $g(n) = n$. On all the even values, $g(n)$ is larger, but on all the odd values, $f(n)$ grows more quickly.

\item
The advantage is that you get the result of part a which is a nice property. A disadvantage is that the infinite set of points on which you are making claims of the behavior could  be very sparse. Also, there is nothing said about the behavior when outside of this infinite set, it can do whatever it wants.

\item
A function $f$ can only be in $\Theta(g(n))$ if $f(n)$ has an infinite tail that is non--negative. In this case, the definition of $O(g(n))$ agrees with $O'(g(n))$. Similarly, for a function to be in $\Omega(g(n))$, we need that $f(n)$ is non-negative for some infinite tail, on which $O(g(n))$ is identical to $O'(g(n))$. So, we have that in both directions, changing $O$ to $O'$ does not change anything.

\item
Suppose $f(n)\in \overset{\sim}{\Theta}(g(n))$, then $\exists c_1,c_2,k_1,k_2,n_0, \forall n\ge n_0, 0\le \frac{c_1 g(n)}{\lg^{k_1}(n)} \le f(n) \le c_2 g(n)\lg^{k_2}(n)$, if we just look at these inequalities separately, we have $\frac{c_1 g(n)}{\lg^{k_1}(n)} \le f(n)$ ($f(n) \in \overset{\sim}{\Omega}(g(n))$) and $f(n) \le c_2 g(n)\lg^{k_2}(n)$ ($f(n)\in \overset{\sim}{O}(g(n))$).

Now for the other direction. Suppose that we had $\exists n_1, c_1,k_1 \forall n\ge n_1, \frac{c_1 g(n)}{\lg^{k_1}(n)} \le f(n)$ and $\exists n_2,c_2,k_2, \forall n\ge n_2, f(n)\le c_2g(n)\lg^{k_2}(n)$. Putting these together, and letting $n_0 = \max(n_1,n_2)$, we have $\forall n\ge n_0, \frac{c_1 g(n)}{\lg^{k_1}(n)} \le f(n) \le c_2 g(n)\lg^{k_2}(n)$. 

\end{enumerate}


\noindent\textbf{Problem 3-6}\\

\begin{tabular}{c|c|c|}
$f(n)$ & $c$ & $f_c^*(n)$ \\ \hline
$n-1$ & 0 & $\lceil n \rceil $ \\ \hline
$\log n$ & 1 & $\log^*n$ \\ \hline
$n/2$ & 1 &  $\lceil \log(n) \rceil $ \\ \hline
$n/2$ & $2$ & $\lceil \log(n) \rceil - 1$ \\ \hline
$\sqrt{n}$ & 2 & $\log \log n$ \\ \hline
$\sqrt{n}$ & 1 & undefined \\ \hline
$n^{1/3}$ & 2 & $\log_3 \log_2(n)$ \\ \hline
$n / \log n$ & 2 & $\Omega\left(\frac{\log n}{\log(\log n)} \right)$ \\ \hline
\end{tabular}\\

\end{document}
\documentclass{article}
\usepackage{fancyhdr}
\usepackage{amsthm}
\usepackage{etoolbox}
\usepackage{verbatim}
\usepackage{enumerate}
\usepackage{amsmath}
\usepackage{algorithmicx}
\usepackage{algorithm}
\usepackage{algpseudocode}
\usepackage{tikz}


	
\pagestyle{fancy}
\title{Chapter 10}
\author{Michelle Bodnar, Andrew Lohr}

\newcounter{curnum}
\setcounter{curnum}{0}

\newtheorem{th1}{Exercise}
\newcommand{\calH}{\mathcal{H}}
\newcommand{\calX}{\mathcal{X}}
\newcommand{\calA}{\mathcal{A}}
\newcommand{\calY}{\mathcal{Y}}

\begin{document}
\maketitle

\noindent\textbf{ Exercise 10.1-1} \\

$
\begin{array}{|c|c|c|}
\hline
4&&\\
\hline
4&1&\\
\hline
4&1&3\\
\hline
4&1&\\
\hline
4&1&8\\
\hline
4&1&\\
\hline
\end{array}
$\\

\noindent\textbf{Exercise 10.1-2}\\

We will call the stacks $T$ and $R$. Initially, set $T.top = 0$ and $R.top = n+1$.  Essentially, stack $T$ uses the first part of the array and stack $R$ uses the last part of the array.  In stack $T$, the top is the rightmost element of $T$.  In stack $R$, the top is the leftmost element of $R$. 
\begin{algorithm}
\caption{PUSH(S,x)}
\begin{algorithmic}[1]
\If{$S == T$}
	\If{$T.top + 1 == R.top$}
		\State \textbf{error} ``overflow''
	\Else
		\State $T.top = T.top + 1$
		\State $T[T.top] = x$
	\EndIf
\EndIf
\If{$S == R$}
	\If{$R.top - 1 == T.top$}
		\State \textbf{error} ``overflow''
	\Else
		\State $R.top = R.top - 1$
		\State $T[T.top] = x$
	\EndIf
\EndIf
\end{algorithmic}
\end{algorithm}

\begin{algorithm}
\caption{POP(S)}
\begin{algorithmic}
\If{$S==T$}
	\If{$T.top == 0$}
		\State \textbf{error} ``underflow''
	\Else
		\State $T.top = T.top - 1$. 
		\State \textbf{return} $T[T.top + 1]$
	\EndIf
\EndIf
\If{$S==R$}
	\If{$R.top == n+1$}
		\State \textbf{error} ``underflow''
	\Else
		\State $R.top = R.top + 1$. 
		\State \textbf{return} $R[R.top - 1]$
	\EndIf
\EndIf
\end{algorithmic}
\end{algorithm}


\noindent\textbf{ Exercise 10.1-3} \\

$
\begin{array}{|c|c|c|c|}
\hline
4&&&\\
\hline
4&1&&\\
\hline
4&1&3&\\
\hline
&1&3&\\
\hline
&1&3&8\\
\hline
&&3&8\\
\hline
\end{array}
$

\noindent\textbf{Exercise 10.1-4}\\

\begin{algorithm}
\caption{ENQUEUE}
\begin{algorithmic}
\If{$Q.head == Q.tail + 1$, or $Q.head == 1$ and $Q.tail == Q.length}$
	\State \textbf{error} ``overflow''
\EndIf
\State $Q[Q.tail] = x$
\If{$Q.tail == Q.length$}
	\State $Q.tail = 1$
\Else
	\State $Q.tail = Q.head + 1$
\EndIf
\end{algorithmic}
\end{algorithm}

\begin{algorithm}
\caption{DEQUEUE}
\begin{algorithmic}
\If{$Q.tail == Q.head$}
	\State \textbf{error} ``underflow''
\EndIf
\State $x = Q[Q.head]$
\If{$Q.head == Q.length$}
	\State $Q.head = 1$
\Else
	\State $Q.head = Q.head + 1$
\EndIf
\State \textbf{return} $x$
\end{algorithmic}
\end{algorithm}

\noindent\textbf{ Exercise 10.1-5} \\

As in the example code given in the section, we will neglect to check for overflow and underflow errors.

\begin{algorithm}
\caption{HEAD-ENQUEUE(Q,x)}
\begin{algorithmic}
\State $Q[Q.head] = x$
\If{$Q.head == 1$}
\State $Q.head = Q.length$
\Else
\State $Q.head = Q.head-1$
\EndIf
\end{algorithmic}
\end{algorithm}

\begin{algorithm}
\caption{TAIL-ENQUEUE(Q,x)}
\begin{algorithmic}
\State $Q[Q.tail] = x$
\If{$Q.tail == Q.length$}
\State $Q.tail = 1$
\Else
\State $Q.tail = Q.tail+1$
\EndIf
\end{algorithmic}
\end{algorithm}

\begin{algorithm}
\caption{HEAD-DEQUEUE(Q,x)}
\begin{algorithmic}
\State $x = Q[Q.head] $
\If{$Q.head == Q.length$}
\State $Q.head = 1$
\Else
\State $Q.head = Q.head+1$
\EndIf
\end{algorithmic}
\end{algorithm}

\begin{algorithm}
\caption{TAIL-DEQUEUE(Q,x)}
\begin{algorithmic}
\State $x = Q[Q.tail]$
\If{$Q.tail == 1$}
\State $Q.tail = Q.length$
\Else
\State $Q.tail = Q.tail-1$
\EndIf
\end{algorithmic}
\end{algorithm}

\noindent\textbf{Exercise 10.1-6}\\

The operation enqueue will be the same as pushing an element on to stack 1.  This operation is $O(1)$. To dequeue, we pop an element from stack 2.  If stack 2 is empty, for each element in stack 1 we pop it off, then push it on to stack 2. Finally, pop the top item from stack 2. This operation is $O(n)$ in the worst case. \\

\noindent\textbf{ Exercise 10.1-7} \\

The following is a way of implementing a stack using two queues, where pop takes linear time, and push takes constant time. The first of these ways, consists of just enqueueing each element as you push it. Then, to do a pop, you dequque each element from one of the queues and place it in the other, but stopping just before the last element. Then, return the single element left in the original queue.\\



\noindent\textbf{ Exercise 10.2-1} \\

To insert an element in constant time, just add it to the head by making it point to the old head and have it be the head. To delete an element, it needs linear time because there is no way to get a pointer to the previous element in the list without starting at the head and scanning along.\\

\noindent\textbf{Exercise 10.2-2}\\

The PUSH(L,x) operation is exactly the same as LIST-INSERT(L,x).  The POP operation sets $x$ equal to $L.head$, calls LIST-DELETE(L,L.head), then returns $x$. \\


\noindent\textbf{ Exercise 10.2-3} \\

In addition to the head, also keep a pointer to the last element in the linked list. To enqueue, insert the element after the last element of the list, and set it to be the new last element. To dequeue, delete the first element of the list and return it.\\

\noindent\textbf{Exercise 10.2-4}\\

First let $L.nil.key = k$.  Then run LIST-SEARCH' as usual, but remove the check that $x \neq L.nil$. \\

\noindent\textbf{ Exercise 10.2-5} \\

To insert, just do list insert before the current head, in constant time. To search, start at the head, check if the element is the current node being inspected, check the next element, and so on until at the end of the list or you found the element. This can take linear time in the worst case. To delete, again linear time is used because there is no way to get to the element immediately before the current element without starting at the head and going along the list.\\

\noindent\textbf{Exercise 10.2-6}\\

Let $L_1$ be a doubly linked list containing the elements of $S_1$ and $L_2$ be a doubly linked list containing the elements of $S_2$.  We implement UNION as follows: Set $L_1.nil.prev.next = L_2.nil.next$ and $L_2.nil.next.prev = L_1.nil.prev$ so that the last element of $L_1$ is followed by the first element of $L_2$.  Then set $L_1.nil.prev = L_2.nil.prev$ and $L_2.nil.prev.next = L_1.nil$, so that $L_1.nil$ is the sentinal for the doubly linked list containing all the elements of $L_1$ and $L_2$. \\

\noindent\textbf{ Exercise 10.2-7} \\

\begin{algorithm}
\caption{REVERSE(L)}
\begin{algorithmic}
\State $a = L.head.next$
\State $b = L.head$
\While{$a\neq NIL$}
\State $tmp = a.next$
\State $a.next = b$
\State $b = a$
\State $a= tmp$
\EndWhile
\State $L.head  = b$
\end{algorithmic}
\end{algorithm}

\noindent\textbf{Exercise 10.2-8}\\

We will store the pointer value for $L.head$ separately, for convenience.  In general, $A$ XOR $(A$ XOR $ C) = C$, so once we know one pointer's true value we can recover all the others (namely $L.head$) by applying this rule. Assuming there are at least two elements in the list, the first element will contain exactly the address of the second. 

\begin{algorithm}
\caption{LISTnp-SEARCH(L,k)}
\begin{algorithmic}
\State $p = NIL$
\State $x=L.head$
\While{$x \neq NIL$ and $x.key \neq k$}
	\State $temp = x$
	\State $x = p XOR x.np$
	\State $p = temp$
\EndWhile
\end{algorithmic}
\end{algorithm}


\begin{algorithm}
\caption{LISTnp-INSERT(L,x)}
\begin{algorithmic}
\State $x.np = L.head$
\State $L.nil.np = x XOR (L.nil.np XOR L.head)$
\State $L.head = x$
\end{algorithmic}
\end{algorithm}

\begin{algorithm}
\caption{LISTnp-Delete(L,x)}
\begin{algorithmic}
\State $L.nil.np = L.nil.np XOR L.head XOR L.head.np$
\State $L.head.np.np = L.head.np.np XOR L.head$
\end{algorithmic}
\end{algorithm}

To reverse the list, we simply need to make the head be the ``last'' element before $L.nil$ instead of the first one after this.  This is done by setting $L.head = L.nil.np XOR L.head$. \\

\noindent\textbf{ Exercise 10.3-1} \\

A multiple array version could be $L=2$,

$
\begin{array}{|c|c|c|c|c|c|c|}
\hline
/&3&4&5&6&7&/\\

\hline
&12&4&8&19&5&11\\

\hline
&/&2&3&4&5&6\\

\hline
\end{array}
$

A single array version could be $L=4$,

$
\begin{array}{|c|c|c|c|c|c|c|c|c|c|c|c|c|c|c|c|c|c|c|c|c|}
\hline
\,\,&\,\,&\,\,&12&7&/&4&10&4&8&13&7&19&16&10&5&19&13&11&/&16\\
\hline
\end{array}
$

\noindent\textbf{Exercise 10.3-2}\\

\begin{algorithm}
\caption{Allocate-Object()}
\begin{algorithmic}
\If{$free == NIL$}
	\State \textbf{error} ``out of space''
\Else
	\State $x = free$
	\State $free = A[x+1]$
\EndIf
\end{algorithmic}
\end{algorithm}

\begin{algorithm}
\caption{Free-Object(x)}
\begin{algorithmic}
\State $A[x+1] = free$
\State $free = x$
\end{algorithmic}
\end{algorithm}

\noindent\textbf{ Exercise 10.3-3} \\

Allocate object just returns the index of some cells that it's guarenteed to not give out again until they've been freed. The prev attribute is not modified because only the next attribute is used by the memory manager, it's up to the code that calls allocate to use the prev and key attributes as it sees fit.\\

\noindent\textbf{Exercise 10.3-4}\\

For ALLOCATE-OBJECT, we will keep track of the next available spot in the array, and it will always be one greater than the number of elements being stored.  For FREE-OBJECT(x), when a space is freed, we will decrement the position of each element in a position greater than that of $x$ by 1 and update pointers accordingly.  This takes linear time.

\noindent\textbf{ Exercise 10.3-5} \\

See the algorithm ${COMPACTIFY-LIST(L,F)}$

\begin{algorithm}
\caption{COMPACTIFY-LIST(L,F)}
\begin{algorithmic}
\If{n=m}
\State \Return
\EndIf
\State $e = \max\{ \max_{i\in[m]} \{|key[i]|\},\max_{i\in L}\{|key[i]|\}\}$
\State increase every element of $key[1..m]$ by $2e$
\State for every element of $L$, if its key is greater than $e$, reduce it by $2e$
\State $f = 1$
\While{$key[f] <e$}
\State $f++$
\EndWhile
\State $a = L.head$
\If{$a>m$}
\State $next[prev[f]] = next[f]$
\State $prev[next[f]] = prev[f]$
\State $next[f] = next[a]$
\State $key[f] = key[a]$
\State $prev[f] = prev[a]$
\State $FREE-OBJECT(a)$
\State $f++$
\While{$key[f] <e$}
\State $f++$
\EndWhile
\EndIf
\While{$a\neq L.head$}
\If{$a>m$}
\State $next[prev[f]] = next[f]$
\State $prev[next[f]] = prev[f]$
\State $next[f] = next[a]$
\State $key[f] = key[a]$
\State $prev[f] = prev[a]$
\State $FREE-OBJECT(a)$
\State $f++$
\While{$key[f] <e$}
\State $f++$
\EndWhile
\EndIf
\EndWhile
\end{algorithmic}
\end{algorithm}

\noindent\textbf{ Exercise 10.4-1} \\

\begin{tikzpicture}[level/.style={sibling distance=50mm/#1}]
\node [circle,draw] (z){18}
  child {node [circle,draw] (a) {12}
    child {node [circle,draw] (b) {7}
      }
    child {node [circle,draw] (g) {4}
	child {node [circle,draw] (c) {5}}
  }}
  child {node [circle,draw] (a) {10}
    child {node [circle,draw] (b) {2}
	}
    child {node [circle,draw] (g) {21}
  }
        };
\end{tikzpicture}

Note that indices 8 and 2 in the array do not appear, and, in fact do not represent a valid tree.\\

\noindent\textbf{Exercise 10.4-2}\\

See the algorithm PRINT-TREE.

\begin{algorithm}
\caption{PRINT-TREE(T.root)}
\begin{algorithmic}
\If{$T.root == NIL$}
	\State \textbf{return}
\Else
	\State Print $T.root.key$
	\State PRINT-TREE(T.root.left)
	\State PRINT-TREE(T.root.right)
\EndIf
\end{algorithmic}
\end{algorithm}

\noindent\textbf{ Exercise 10.4-3} \\
\begin{algorithm}
\caption{INORDER-PRINT(T)}
\begin{algorithmic}
\State let S be an empty stack
\State $push(S,T)$
\While{S is not empty}
\State $U = pop(S)$
\If{$U\neq NIL$}
\State print $U.key$
\State $push(S,U.left)$
\State $push(S,U.right)$
\EndIf
\EndWhile
\end{algorithmic}
\end{algorithm}

\noindent\textbf{Exercise 10.4-4}\\

See the algorithm PRINT-TREE.

\begin{algorithm}
\caption{PRINT-TREE(T.root)}
\begin{algorithmic}
\If{$T.root == NIL$}
	\State \textbf{return}
\Else
	\State Print $T.root.key$
	\State $x = T.root.left-child$
	\While{$x \neq NIL$}
		\State PRINT-TREE(x)
		\State $x = x.right-sibling$
	\EndWhile
\EndIf
\end{algorithmic}
\end{algorithm}

\noindent\textbf{ Exercise 10.4-5} \\
See the algorithm INORDER-PRINT'(T)
\begin{algorithm}
\caption{INORDER-PRINT'(T)}
\begin{algorithmic}
\State $a = T.left$
\State $prev = T$
\While{$a\neq T$}
\If{$prev = a.left$}
\State print $a.key$
\State $prev = a$
\State $a = a.right$
\ElsIf{$prev = a.right$}
\State $prev = a$
\State $a = a.p$
\ElsIf{$prev = a.p$}
\State $prev = a$
\State $a = a.left$
\EndIf
\EndWhile
\State print $T.key$
\State $a = T.right$
\While{$a\neq T$}
\If{$prev = a.left$}
\State print $a.key$
\State $prev = a$
\State $a = a.right$
\ElsIf{$prev = a.right$}
\State $prev = a$
\State $a = a.p$
\ElsIf{$prev = a.p$}
\State $prev = a$
\State $a = a.left$
\EndIf
\EndWhile
\end{algorithmic}
\end{algorithm}

\noindent\textbf{Exercise 10.4-6}\\

Our two pointers will be $left$ and $right$.  For a node $x$, $x.left$ will point to the leftmost child of $x$ and $x.right$ will point to the sibling of $x$ immediately to its right, if it has one, and the parent of $x$ otherwise.  Our boolean value $b$, stored at $x$, will be such that $b = depth(x) \mod 2$.  To reach the parent of a node, simply keep following the ``right'' pointers until the parity of the boolean value changes.  To find all the children of a node, start by finding $x.left$, then follow the ``right'' pointers until the parity of the boolean value changes, ignoring this last node since it will be $x$.

\noindent\textbf{ Problem 10-1} \\

For each, we assume sorted means sorted in ascending order

$
\begin{array}{c|c|c|c|c|}
&unsorted,single& sorted,single&unsorted,double&sorted,double\\
\hline
SEARCH(L,k)&n&n&n&n\\
\hline
INSERT(L,x)&1&1&1&1\\
\hline
DELETE(L,x)&n&n&1&1\\
\hline
SUCCESSOR(L,x)&n&1&n&1\\
\hline
PREDECESSOR(L,x)&n&n&n&1\\
\hline
MINIMUM(L,x)&n&1&n&1\\
\hline
MAXIMUM(L,x)&n&n&n&1\\
\hline

\end{array}
$

\noindent\textbf{ Problem 10-2} \\

%Not done

\noindent\textbf{ Problem 10-3} \\

\begin{enumerate}[a.]
\item
If the original version of the algorithm takes only t iterations, then, we have that it was only at most t random skips though the list to get to the desired value, since each iteration of the original while loop is a possible random jump followed by a normal step through the linked list.

\item
The for loop on lines 2-7 will get run exactly t times, each of which is constant runtime. After that, the while loop on lines 8-9 will be run exactly $X_t$ times. So, the total runtime is $O(t+E[X_t])$.

\item

Using equation C.25, we have that $E[X_t] = \sum_{i=1}^{\infty} Pr(X_t\ge i)$. So, we need to show that $ Pr(X_t\ge i) \le (1-i/n)^t$. This can be seen because having $X_t$ being greater than $i$ means that each random choice will result in an element that is either at least $i$ steps before the desitred element, or is after the desired element. There are $n-i$ such elements, out of the total $n$ elements that we were pricking from. So, for a single one of the choices to be from such a range, we have a probability of $(n-i)/n = (1-i/n)$. Since each of the selections was independent, the total probability that all of them were is $(1-i/n)^t$, as desired. Lastle, we can note that since the linked list has length $n$, the probability that $X_t$ is greater than $n$ is equal to zero.

\item

Since we have that $t>0$, we know that the function $f(x) = x^t$ is increasing, so, that means that $\lfloor x \rfloor ^t \le f(x)$. So,
\[
\sum_{r=0}^{n-1} r^t  = \int_{0}^n \lfloor r \rfloor^t dr \le \int_0^n f(r) dr = \frac{n^{t+1}}{t+1}
\]

\item

\[
E[X_t] \le \sum_{r=1}^n (1-r/n)^t = \sum_{r=1}^n \sum_{i=0}^t \binom{t}{i} (-r/n)^i = \sum_{i=0}^t \sum_{r=1}^n  \binom{t}{i} (-r/n)^i \]\[
= \sum_{i=0}^t  \binom{t}{i} (-1)^i\left(n^i -1+ \sum_{r=0}^{n-1} (r)^t\right)/n \le \sum_{i=0}^t  \binom{t}{i} (-1)^i\left( n^i-1+ \frac{n^{i+1}}{i+1}\right)/n 
\]
\[
\le  \sum_{i=0}^t  \binom{t}{i} (-1)^i \frac{n^i}{i+1} =  \frac{1}{t+1} \sum_{i=0}^t \binom{t+1}{i+1} (-n)^i \le \frac{(1-n)^{t+1}}{t+1}
\]

\item

We just put together parts b and e to get that it runs in time $O(t+n/(t+1))$. But, this is the same as $O(t+n/t)$.

\item

Since we have that for any number of iterations $t$ that the first algorithm takes to find its answer, the second algorithm will return it in time $O(t+ n/t)$. In particular, if we just have that $t = \sqrt{n}$. The second algorithm takes time only $O(\sqrt{n})$. This means that tihe first list search algorithm is $O(\sqrt{n})$ as well.

\item

%If the list contains repeated key values, then, we may randomly skip ahead at some point, then, jump backwards to a key that had the same value, but was further back in the list. The analysis will break down when we try to bound the probability that $X_t$ is greater than $i$. This is because we don't need that every single random choice was at least $i$ away from the correct position, just could of been to any element whoose key was the same as that spot as well.

if we don't have distinct key values, then, we may randomly select an element that is further along than we had been before, but not jump to it because it has the same key as what we were currently at. The analysis will break when we try to bound the probability that $X_t \ge i$.

\end{enumerate}

\end{document} 